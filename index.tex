% Options for packages loaded elsewhere
% Options for packages loaded elsewhere
\PassOptionsToPackage{unicode}{hyperref}
\PassOptionsToPackage{hyphens}{url}
\PassOptionsToPackage{dvipsnames,svgnames,x11names}{xcolor}
%
\documentclass[
  letterpaper,
  DIV=11,
  numbers=noendperiod]{scrreprt}
\usepackage{xcolor}
\usepackage{amsmath,amssymb}
\setcounter{secnumdepth}{5}
\usepackage{iftex}
\ifPDFTeX
  \usepackage[T1]{fontenc}
  \usepackage[utf8]{inputenc}
  \usepackage{textcomp} % provide euro and other symbols
\else % if luatex or xetex
  \usepackage{unicode-math} % this also loads fontspec
  \defaultfontfeatures{Scale=MatchLowercase}
  \defaultfontfeatures[\rmfamily]{Ligatures=TeX,Scale=1}
\fi
\usepackage{lmodern}
\ifPDFTeX\else
  % xetex/luatex font selection
\fi
% Use upquote if available, for straight quotes in verbatim environments
\IfFileExists{upquote.sty}{\usepackage{upquote}}{}
\IfFileExists{microtype.sty}{% use microtype if available
  \usepackage[]{microtype}
  \UseMicrotypeSet[protrusion]{basicmath} % disable protrusion for tt fonts
}{}
\makeatletter
\@ifundefined{KOMAClassName}{% if non-KOMA class
  \IfFileExists{parskip.sty}{%
    \usepackage{parskip}
  }{% else
    \setlength{\parindent}{0pt}
    \setlength{\parskip}{6pt plus 2pt minus 1pt}}
}{% if KOMA class
  \KOMAoptions{parskip=half}}
\makeatother
% Make \paragraph and \subparagraph free-standing
\makeatletter
\ifx\paragraph\undefined\else
  \let\oldparagraph\paragraph
  \renewcommand{\paragraph}{
    \@ifstar
      \xxxParagraphStar
      \xxxParagraphNoStar
  }
  \newcommand{\xxxParagraphStar}[1]{\oldparagraph*{#1}\mbox{}}
  \newcommand{\xxxParagraphNoStar}[1]{\oldparagraph{#1}\mbox{}}
\fi
\ifx\subparagraph\undefined\else
  \let\oldsubparagraph\subparagraph
  \renewcommand{\subparagraph}{
    \@ifstar
      \xxxSubParagraphStar
      \xxxSubParagraphNoStar
  }
  \newcommand{\xxxSubParagraphStar}[1]{\oldsubparagraph*{#1}\mbox{}}
  \newcommand{\xxxSubParagraphNoStar}[1]{\oldsubparagraph{#1}\mbox{}}
\fi
\makeatother


\usepackage{longtable,booktabs,array}
\usepackage{calc} % for calculating minipage widths
% Correct order of tables after \paragraph or \subparagraph
\usepackage{etoolbox}
\makeatletter
\patchcmd\longtable{\par}{\if@noskipsec\mbox{}\fi\par}{}{}
\makeatother
% Allow footnotes in longtable head/foot
\IfFileExists{footnotehyper.sty}{\usepackage{footnotehyper}}{\usepackage{footnote}}
\makesavenoteenv{longtable}
\usepackage{graphicx}
\makeatletter
\newsavebox\pandoc@box
\newcommand*\pandocbounded[1]{% scales image to fit in text height/width
  \sbox\pandoc@box{#1}%
  \Gscale@div\@tempa{\textheight}{\dimexpr\ht\pandoc@box+\dp\pandoc@box\relax}%
  \Gscale@div\@tempb{\linewidth}{\wd\pandoc@box}%
  \ifdim\@tempb\p@<\@tempa\p@\let\@tempa\@tempb\fi% select the smaller of both
  \ifdim\@tempa\p@<\p@\scalebox{\@tempa}{\usebox\pandoc@box}%
  \else\usebox{\pandoc@box}%
  \fi%
}
% Set default figure placement to htbp
\def\fps@figure{htbp}
\makeatother





\setlength{\emergencystretch}{3em} % prevent overfull lines

\providecommand{\tightlist}{%
  \setlength{\itemsep}{0pt}\setlength{\parskip}{0pt}}



 


\KOMAoption{captions}{tableheading}
\makeatletter
\@ifpackageloaded{bookmark}{}{\usepackage{bookmark}}
\makeatother
\makeatletter
\@ifpackageloaded{caption}{}{\usepackage{caption}}
\AtBeginDocument{%
\ifdefined\contentsname
  \renewcommand*\contentsname{Table of contents}
\else
  \newcommand\contentsname{Table of contents}
\fi
\ifdefined\listfigurename
  \renewcommand*\listfigurename{List of Figures}
\else
  \newcommand\listfigurename{List of Figures}
\fi
\ifdefined\listtablename
  \renewcommand*\listtablename{List of Tables}
\else
  \newcommand\listtablename{List of Tables}
\fi
\ifdefined\figurename
  \renewcommand*\figurename{Figure}
\else
  \newcommand\figurename{Figure}
\fi
\ifdefined\tablename
  \renewcommand*\tablename{Table}
\else
  \newcommand\tablename{Table}
\fi
}
\@ifpackageloaded{float}{}{\usepackage{float}}
\floatstyle{ruled}
\@ifundefined{c@chapter}{\newfloat{codelisting}{h}{lop}}{\newfloat{codelisting}{h}{lop}[chapter]}
\floatname{codelisting}{Listing}
\newcommand*\listoflistings{\listof{codelisting}{List of Listings}}
\makeatother
\makeatletter
\makeatother
\makeatletter
\@ifpackageloaded{caption}{}{\usepackage{caption}}
\@ifpackageloaded{subcaption}{}{\usepackage{subcaption}}
\makeatother
\usepackage{bookmark}
\IfFileExists{xurl.sty}{\usepackage{xurl}}{} % add URL line breaks if available
\urlstyle{same}
\hypersetup{
  pdftitle={Alfaza Naufal Zakiy},
  pdfauthor={18222126 Alfaza Naufal Zakiy},
  colorlinks=true,
  linkcolor={blue},
  filecolor={Maroon},
  citecolor={Blue},
  urlcolor={Blue},
  pdfcreator={LaTeX via pandoc}}


\title{Alfaza Naufal Zakiy}
\usepackage{etoolbox}
\makeatletter
\providecommand{\subtitle}[1]{% add subtitle to \maketitle
  \apptocmd{\@title}{\par {\large #1 \par}}{}{}
}
\makeatother
\subtitle{Portfolio Asesmen II-2100 KIPP}
\author{18222126 Alfaza Naufal Zakiy}
\date{2025-09-15}
\begin{document}
\maketitle

\renewcommand*\contentsname{Table of contents}
{
\hypersetup{linkcolor=}
\setcounter{tocdepth}{2}
\tableofcontents
}

\bookmarksetup{startatroot}

\chapter*{Selamat Datang}\label{selamat-datang}
\addcontentsline{toc}{chapter}{Selamat Datang}

\markboth{Selamat Datang}{Selamat Datang}

Saya Alfaza Naufal Zakiy (18222126), mahasiswa Program Studi Sistem dan
Teknologi Informasi, Sekolah Teknik Elektro dan Informatika -- Institut
Teknologi Bandung (ITB).

Website ini dibuat sebagai bagian dari asesmen mata kuliah II2100
Komunikasi Interpersonal dan Publik (KIPP) yang berfokus pada
pengembangan kemampuan berkomunikasi secara efektif, reflektif, dan
bermakna dalam konteks akademik maupun profesional.

Melalui laman ini, saya membagikan proses belajar, refleksi diri, serta
karya-karya yang menggambarkan perjalanan saya dalam memahami esensi
komunikasi interpersonal---mulai dari kesadaran diri, empati, hingga
kemampuan menyampaikan gagasan secara terbuka dan konstruktif.

Setiap bagian di website ini dirancang untuk menunjukkan bahwa
komunikasi bukan sekadar keterampilan berbicara, tetapi juga seni
membangun hubungan dan menciptakan pemahaman bersama melalui setiap kata
dan tindakan.

\bookmarksetup{startatroot}

\chapter{UTS-1 All About Me}\label{uts-1-all-about-me}

\begin{figure}[H]

{\centering \includegraphics[width=9.5\linewidth,height=\textheight,keepaspectratio]{All_About_me/../images/AZRL.png}

}

\caption{About Me}

\end{figure}%

Halo semuannya salam kenal 😊, Perkenalkan Nama saya Alfaza Naufal
Zakiy, teman-teman biasanya memanggil saya Zakiy, Zek, atau kadang Jack.
Saya mahasiswa Sistem dan Teknologi Informasi di ITB. Saya tipe orang
yang selalu ingin tahu bagaimana sesuatu bekerja. Sering kali saya lebih
sibuk berpikir dan mengamati daripada ikut obrolan yang tidak jelas
arahnya. Saya merasa itu yang membuat saya berbeda, karena saya suka
mencari makna di balik hal-hal kecil.

Sekarang fokus utama saya adalah belajar. Tidak hanya belajar akademik
di kampus, tapi juga belajar memahami diri saya sendiri. Saya lebih suka
membandingkan diri saya hari ini dengan diri saya kemarin daripada
membandingkan diri dengan orang lain. Dari situ saya bisa melihat
kemajuan kecil yang mungkin tidak terlihat oleh orang lain, tapi berarti
banyak buat saya.

Dalam berkomunikasi, saya cukup empatik, suka mendengarkan, dan terbiasa
menyampaikan sesuatu dengan padat dan jelas. Saat ini saya sedang
berlatih untuk berbicara lebih percaya diri, membaca bahasa tubuh orang
lain, dan memahami makna yang tidak selalu diucapkan secara langsung. Di
luar itu, saya juga sedang belajar mengatur waktu dengan lebih baik,
karena sering kali saya menyesal setelah menyadari waktu saya terbuang
untuk hal yang kurang penting. Nilai komunikasi yang paling saya pegang
adalah empati, kejujuran, dan kejelasan.

\section{Momen dan Pelajaran}\label{momen-dan-pelajaran}

Saya pernah ada di situasi di mana saya takut bertemu teman sekelas
sendiri karena sempat melakukan kesalahan. Waktu itu saya memilih untuk
menghindar. Hasilnya, memang terasa lega sesaat, tapi saya juga
kehilangan kesempatan untuk memperbaiki hubungan. Dari situ saya belajar
bahwa menghindar tidak pernah menyelesaikan apa pun. Kalau salah, lebih
baik diakui dan dihadapi. Rasa malu biasanya hanya di awal, tapi rasa
lega datang setelahnya.

Ada juga momen ketika ekspektasi saya terhadap teman tidak sesuai dengan
kenyataan. Saya mencoba mengalihkan pikiran dengan bekerja di depan
laptop, tapi perasaan sedih tetap datang. Akhirnya saya menangis selama
lima menit, bukan karena lemah, tapi karena terlalu lelah menahan
semuanya sendiri. Dari situ saya belajar bahwa konflik yang dibiarkan
berlarut-larut hanya membuat capek. Kadang yang kita butuh bukan
pembenaran, tapi penerimaan bahwa keadaan memang tidak selalu sesuai
harapan.

Pernah juga saya menghadapi situasi saat harus mengerjakan UTS take home
dan waktu hampir habis. Saya panik, bekerja dengan terburu-buru, dan
akhirnya banyak fitur yang tidak berjalan. Dari situ saya belajar
pentingnya memprioritaskan tugas. Sekarang saya terbiasa memilah
pekerjaan dengan bantuan prinsip Urgent-Important Matrix agar bisa fokus
pada hal yang benar-benar penting.

Setelah ikut mata kuliah KIPP, cara pandang saya terhadap komunikasi
berubah cukup banyak. Saya mulai sadar bahwa sikap, ekspresi, dan bahasa
tubuh punya pengaruh besar. Sekarang saya lebih berusaha memahami makna
dari apa yang orang lain sampaikan, melihatnya dari berbagai sudut
pandang, bukan hanya dari sudut pandang saya sendiri.

\section{Presentasi dan Cara Bicara}\label{presentasi-dan-cara-bicara}

Dulu saya takut sekali presentasi. Saya tipe orang yang perlu waktu
berpikir panjang sebelum bicara, jadi kalau ada pertanyaan sulit, saya
sering diam cukup lama. Saat itu saya suka membaca slide karena isinya
terlalu penuh. Sekarang saya belajar tampil lebih ringan. Saya hanya
menuliskan poin penting dan benar-benar memahami materi sebelum
presentasi, supaya penjelasannya tetap mengalir.

Pesan utama favorit saya saat presentasi adalah ``Jalanin aja dulu.''
Buat saya, kalimat ini sederhana tapi bermakna dalam. Tidak perlu
menunggu sempurna untuk mulai, yang penting berani mencoba. Kesempurnaan
akan datang lewat prosesnya. Pesan ini saya gunakan juga untuk
mengingatkan diri sendiri agar tidak menunda.

Untuk melakukan pembukaan presentasi yang menarik dan fresh, saya suka
memakai kalimat kontras yang sederhana. Pernah saya membuka presentasi
dengan kalimat ``Banyak orang takut gagal, tapi saya percaya mending
menyesal mencoba daripada menyesal tidak mencoba.'' Biasanya audiens
langsung senyum dan fokus mendengarkan.

Saya bukan orang yang suka melucu di depan umum, jadi cara saya
mencairkan suasana biasanya lewat senyum, nada bicara yang hangat, dan
kontak mata. Kadang saya menambahkan cerita singkat yang relevan supaya
audiens tetap merasa dekat tanpa harus tertawa.

\section{Prinsip dan Refleksi}\label{prinsip-dan-refleksi}

Ada tiga prinsip utama yang selalu saya pegang, baik di kelas maupun di
luar kelas. Pertama, dengarkan dulu sebelum berbicara. Kedua, jelaskan
niat sebelum memberi saran. Ketiga, sampaikan hal sulit dengan tenang
dan jelas.

Menurut saya, komunikasi itu seperti jaringan data. Kalau sinyalnya
buruk, pesan bisa salah diterima. Maka saya berusaha memastikan
``jaringan'' antara saya dan orang lain tetap stabil dengan cara
mendengarkan baik-baik, mengklarifikasi, dan memastikan pesannya
tersampaikan dengan jelas.

Saya juga berusaha menyeimbangkan antara fakta dan empati. Misalnya,
ketika teman saya melakukan kesalahan, saya tidak langsung bilang ``kamu
salah'', tapi saya mulai dengan kalimat ``Aku ngerti maksudmu, tapi
mungkin di bagian ini bisa kita cek lagi datanya.'' Dengan begitu, pesan
bisa sampai tanpa membuat orang merasa diserang.

Kalimat yang paling bermakna buat saya adalah ``Remember why you
started.'' Setiap kali saya merasa lelah atau ingin menyerah, kalimat
itu selalu mengingatkan saya untuk kembali pada alasan awal saya
memulai. Bukan untuk menjadi sempurna, tapi untuk menjadi lebih baik
dari kemarin.

\section{Penutup}\label{penutup}

Akhirnya kita masuk di bagian paling akhir di bagian UTS-1 All About Me.
Kalau saya boleh menyimpulkan perjalanan saya di mata kuliah KIPP,
mungkin satu kata yang paling pas adalah ``berproses.'' Saya belajar
bahwa komunikasi bukan cuma soal berbicara, tapi juga tentang bagaimana
kita hadir sepenuhnya di hadapan orang lain. Kadang cukup dengan
mendengarkan, tersenyum, atau jujur mengungkapkan perasaan kita.

Melalui tugas ini, saya ingin berbagi perjalanan kecil saya sebagai
seseorang yang dulunya sering overthinking sebelum bicara, dan kini
sedang belajar untuk lebih terbuka serta berani menyampaikan pikiran
dengan tulus.

Terima kasih sudah membaca cerita ini. Kalau ada masukan, saya akan
sangat senang mendengarnya. Karena bagi saya, komunikasi yang baik tidak
berhenti di kata-kata yang diucapkan, tapi terus hidup dari apa yang
kita dengar setelahnya.

\bookmarksetup{startatroot}

\chapter{UTS-2 My Songs for You}\label{uts-2-my-songs-for-you}

Post Wedding Kawah Putih Lirik by Armein Z. R. Langi Music: SUNO

\url{https://youtu.be/KWthEImJ9mY?si=iUV8Ghhj0R3gKQyZ}, with lyrics

Engkau {[}Intro{]} {[}verse1{]} Mereka bilang perlu Ribuan kata untuk
menyatakan cinta. Tapi satu-satunya kata yang kutemukan adalah:
engkau\ldots{}

{[}Verse 2{]} Hidup bergerak begitu deras Siang malam datang dan pergi.
Mereka bilang semua itu akan melunturkan cinta kita\ldots{}

{[}Chorus 1{]} Tetapi engkau masih di sini, dengan tatapan cinta. Engkau
selalu di sini dengan hati penuh kasih.

{[}Chorus 2{]} Tahun demi tahun berlalu Cintamu menjadikan semua ini
begitu indah, Begitu bermakna. Begitu menggairahkan.

{[}Verse4{]} Peristiwa menimpa seperti badai tercurah dari langit.
Menerjang seperti seretan sungai. Melemparku ke dalam dan ke luar, ke
atas dan ke bawah, menghanyutkanku gelagapan ke samudera raya\ldots{}

{[}Chorus 1{]} Tetapi engkau masih di sini, dengan tatapan cinta. Engkau
selalu di sini dengan hati penuh kasih.

{[}Chorus 2{]} Tahun demi tahun berlalu Cintamu menjadikan semua ini
begitu indah, Begitu bermakna. Begitu menggairahkan.

{[}Coda 1{]} Sadarkah engkau bahwa engkau telah menemukan hukum baru
tentang cinta?

{[}Coda 2{]} Bahwa satu tambah satu sama dengan satu Satu kali satu
adalah tak hingga? Dua kurang satu sama dengan kosong. Segala sesuatu
tanpa engkau adalah tiada\ldots?

{[}Chorus 1{]} Tetapi engkau masih di sini, dengan tatapan cinta. Engkau
selalu di sini dengan hati penuh kasih.

{[}Chorus 2{]} Tahun demi tahun berlalu Cintamu menjadikan semua ini
begitu indah, Begitu bermakna. Begitu menggairahkan. {[}Outro{]}

Why am I singing for you? \href{./Rivers\%20In\%20My\%20Mind.mp3}{River
on my Mind} , with lyrics

River on my mind {[}Verse{]} I do not know what has happened with your
life I do not know how much you have suffered What I know is how to be
me And I am still learning

{[}Chorus{]} I hope you find my song beautifully caring for you I hope
you find my song beautifully caring for you

{[}Verse{]} You deserve a caring heart And all of the most beautiful
parts of this world You deserve a gentle touch And a soft place to land
when you fall

{[}Chorus{]} I hope you find my song beautifully caring for you I hope
you find my song beautifully caring for you

{[}Bridge{]} I hope you find my song beautifully caring for you I hope
you find my song beautifully caring for you I hope you find my song
beautifully caring for you I hope you find my song beautifully caring
for you

Falling in love everyday \href{./Heaven\%20on\%20Earth.mp3}{Heaven on
Earth}, with lyrics

Heaven on Earth {[}Verse{]} The breeze whispers soft to the sea A rhythm
flows like it's meant to be Every step feels lighter Free

{[}Verse 2{]} The sunlight dances on your skin Every moment a new song
begins Even the stars seem to lean in fresh

{[}Chorus{]} Fall in love Fall in love Every day Let your heart find a
brand-new way Heaven's here It's not far above Life is beautiful when
you love

{[}Verse 2{]} The sunlight dances on your skin Every moment a new song
begins Even the stars seem to lean in

{[}Prechorus{]} Close your eyes Let it start Feel the world inside your
heart

{[}Chorus{]} Fall in love Fall in love Every day Let your heart find a
brand-new way Heaven's here It's not far above Life is beautiful when
you love

{[}Bridge{]} Catch the rain as it falls on your face Hear the laughter
It fills every space Time is fleeting Don't let it erase This heaven we
chase

\bookmarksetup{startatroot}

\chapter{UTS-3 My Stories for You}\label{uts-3-my-stories-for-you}

A story about my oldest daughter
\href{https://azrl.wordpress.com/2020/07/18/gaun-pengantin-gladys/\#comment-28004}{Gaun
Pemngantin Gladys}

A message to my daughter
\href{https://azrl.wordpress.com/2021/10/06/the-child-who-learned-to-walk-at-the-disneyland/}{The
Child Who Learned to Walk at the Disneyland}

A story for my students
\href{https://azrl.wordpress.com/2008/04/21/fly-my-eagle-fly/}{Fly Eagle
Fly}

A (true) story for my teachers
\href{\%3Chttps://azrl.wordpress.com/2012/11/28/perginya-sang-mahaputera-dan-mahaguru-berkemeja-putih/}{Sang
Mahaguru, Sang Mahaputera}

Teasing story \url{https://www.youtube.com/watch?v=Dg_4PbBlBf4}

\bookmarksetup{startatroot}

\chapter{UTS-4 My SHAPE (Spiritual Gifts, Heart, Abilities, Personality,
Experiences)}\label{uts-4-my-shape-spiritual-gifts-heart-abilities-personality-experiences}

\begin{quote}
\textbf{Tujuan:} Merangkum rancangan diri (charter) agar saya melayani,
berkarya, dan memimpin secara paling selaras dengan karunia dan
pengalaman hidup saya. Dapat langsung ditempel ke halaman \textbf{UTS-4
--- My SHAPE} dan dipakai sebagai acuan aksi 90 hari.
\end{quote}

\section{\texorpdfstring{Sumber
\href{StrengthsProfile-Armein-Langi.pdf}{VIA
assessment}}{Sumber VIA assessment}}\label{sumber-via-assessment}

\section{0) Ringkasan 1 Halaman}\label{ringkasan-1-halaman}

\textbf{Peran Inti:} Profesor \& Elder --- perancang ekosistem
belajar-bernilai, pembimbing, dan pemimpin pelayanan komunitas.
\textbf{Misi:} Mengangkat kualitas hidup melalui \emph{smart
engineering} dan \emph{value-oriented education}, khususnya bagi
lansia/keluarga/komunitas (GRACE), serta pelayanan gerejawi yang
menumbuhkan kasih dan pengharapan. \textbf{Kekuatan Utama:} mengkonsep
sistem utuh, menulis \& mengajar, membangun jejaring, merancang
rubrik/alat evaluasi, menggerakkan proyek lintas-disiplin.
\textbf{Dampak yang Dituju:} karya, kurikulum, dan pelayanan yang
menumbuhkan karakter, keterampilan, serta kesejahteraan berkeadilan.

\textbf{Peta SHAPE (singkat):}

\begin{itemize}
\tightlist
\item
  \textbf{S --- Spiritual Gifts:} Teaching, Shepherding/Pastoring,
  Leadership, Wisdom/Discernment, Exhortation/Encouragement,
  Administration.
\item
  \textbf{H --- Heart (Minat \& Cinta Pelayanan):} pendidikan
  berorientasi nilai; kesejahteraan lansia \& keluarga (GRACE);
  pembinaan iman; menulis kisah/novel/lirik; rekayasa cerdas \& AI untuk
  kebaikan bersama; mentorship mahasiswa-dosen; penguatan jemaat.
\item
  \textbf{A --- Abilities (Kemampuan):} perancangan sistem (PSKVE/TISE),
  kurikulum \& rubrik, riset \& publikasi, menulis multi-format
  (Quarto/LaTeX), pemrograman (Python/R/Prolog/Modelica), komunikasi
  publik, memimpin kolaborasi.
\item
  \textbf{P --- Personality (Gaya Kepribadian Kerja):} strategis \&
  reflektif, berorientasi visi \& nilai, analitis-sistemik, kolaboratif,
  tenang dalam krisis, suka membangun standar \& alat.
\item
  \textbf{E --- Experiences (Pengalaman Kunci):} dosen \& peneliti
  lintas proyek (GRACE, Smart Engineering, pendidikan), Elder \&
  pengorganisasi jemaat, penulis kreatif, arsitek sistem pengetahuan
  (Obsidian/GitHub/Quarto), penggerak sarasehan \& penggalangan
  dukungan.
\end{itemize}

\begin{center}\rule{0.5\linewidth}{0.5pt}\end{center}

\section{1) S --- Spiritual Gifts (Karunia
Rohani)}\label{s-spiritual-gifts-karunia-rohani}

\begin{itemize}
\tightlist
\item
  \textbf{Teaching \& Wisdom/Discernment:} mengubah konsep kompleks
  menjadi peta belajar, rubrik, dan alat evaluasi yang memampukan.
\item
  \textbf{Shepherding/Pastoring \& Exhortation:} membimbing
  individu/kelompok dengan empati, meneguhkan, dan memberi arah.
\item
  \textbf{Leadership \& Administration:} merancang ekosistem
  (orang--proses--alat) dengan target berdampak dan terukur.
\end{itemize}

\textbf{Indikator Bukti:} silabus \& rubrik (II-2100/EL2007), naskah
pengajaran, bimbingan riset, modul/website kelas, program jemaat.

\begin{center}\rule{0.5\linewidth}{0.5pt}\end{center}

\section{2) H --- Heart (Minat, Nilai,
Kepedulian)}\label{h-heart-minat-nilai-kepedulian}

\begin{itemize}
\tightlist
\item
  Pendidikan yang \textbf{mencipta nilai} (CPMK↔rubrik↔artefak nyata).
\item
  \textbf{GRACE}: kualitas hidup lansia/keluarga melalui sistem dukung
  cerdas \& komunitas saling-melayani.
\item
  \textbf{Gereja \& Komunitas}: penguatan iman, kesalingan, dan
  pelayanan kasih.
\item
  \textbf{Kreativitas naratif}: kisah/novel/lirik sebagai sarana edukasi
  \& pengharapan.
\item
  \textbf{Rekayasa cerdas \& AI} untuk kemaslahatan.
\end{itemize}

\textbf{Masalah yang ingin dipecahkan:} kesenjangan antara
pengetahuan--karakter--aksi; pembelajaran kurang bermakna; layanan
komunitas belum terukur dampaknya.

\begin{center}\rule{0.5\linewidth}{0.5pt}\end{center}

\section{3) A --- Abilities (Kemampuan
Andal)}\label{a-abilities-kemampuan-andal}

\begin{itemize}
\tightlist
\item
  \textbf{Perancangan sistem} (PSKVE/TISE), \emph{value co‑creation},
  finansial rekayasa, desain instrumen penilaian.
\item
  \textbf{Kurikulum \& pedagogi}: CPMK↔rubrik↔tugas↔bukti; otomasi alur
  kerja (Python/Quarto/GitHub).
\item
  \textbf{Riset \& penulisan ilmiah}; \textbf{karya kreatif} (prosa,
  lirik, ceramah/khotbah).
\item
  \textbf{Teknis}: Python, R, Prolog (ontologi), Modelica, Quarto/LaTeX,
  Obsidian, GitHub, Graphviz.
\item
  \textbf{Komunikasi \& kepemimpinan}: orasi publik, fasilitasi
  sarasehan, negosiasi kolaborasi.
\end{itemize}

\begin{center}\rule{0.5\linewidth}{0.5pt}\end{center}

\section{4) P --- Personality (Gaya Kerja \&
Kolaborasi)}\label{p-personality-gaya-kerja-kolaborasi}

\begin{itemize}
\tightlist
\item
  \textbf{Strategis‑sistemik} (melihat gambaran besar, memetakan
  bagian-bagian).
\item
  \textbf{Reflektif \& nilai‑driven} (standar etis \& mutu).
\item
  \textbf{Kolaboratif} (membangun jejaring, memberi ruang tumbuh).
\item
  \textbf{Tenang‑tangguh} (fokus hasil jangka panjang).
\item
  \textbf{Pembelajar \& pembuat alat} (suka membuat template, rubrik,
  pipeline).
\end{itemize}

\begin{center}\rule{0.5\linewidth}{0.5pt}\end{center}

\section{5) E --- Experiences (Pengalaman
Pembentuk)}\label{e-experiences-pengalaman-pembentuk}

\begin{itemize}
\tightlist
\item
  \textbf{Akademik \& Riset:} merancang mata kuliah, SLR AI \&
  transformasi digital, proyek GRACE \& Smart Engineering.
\item
  \textbf{Pelayanan \& Organisasi:} Elder GKI, fasilitator sarasehan,
  penggalangan dukungan jemaat, pembinaan rohani.
\item
  \textbf{Kreasi Konten:} penulisan novel/khotbah/lirik; produksi materi
  ajar multi‑format.
\item
  \textbf{Infrastruktur Pengetahuan:} Obsidian--GitHub--Quarto, rubrik
  otomatis, bank soal.
\end{itemize}

\textbf{Pelajaran Inti:} integrasi iman--ilmu--nilai; sistem yang baik
melipatgandakan orang baik; narasi menggerakkan aksi.

\begin{center}\rule{0.5\linewidth}{0.5pt}\end{center}

\section{6) Piagam Diri (Self‑Charter)}\label{piagam-diri-selfcharter}

\textbf{Misi Hidup:} merancang dan menggerakkan ekosistem pembelajaran
\& pelayanan yang memerdekakan, bermakna, dan berkeadilan. \textbf{Nilai
Inti:} kasih, integritas, kebijaksanaan, keberanian, mutu, keberpihakan
pada yang lemah. \textbf{Peran Inti:} Perancang sistem
nilai‑pembelajaran; Pembimbing \& pengajar; Pemimpin pelayanan
komunitas. \textbf{Kompas Keputusan:} (1) Dampak pada manusia; (2)
Keselarasan nilai; (3) Keberlanjutan; (4) Kemampuan tim mengelola; (5)
Bukti terukur. \textbf{Janji Pelayanan:} hadir dengan empati, mendengar,
memberi arah praktis, membangun alat agar orang lain bertumbuh.
\textbf{Batasan:} menolak proyek yang mengabaikan martabat
manusia/etika; menjaga ritme kerja‑istirahat‑keluarga.

\begin{center}\rule{0.5\linewidth}{0.5pt}\end{center}

\section{7) Narasi 90 Detik (Elevator
Pitch)}\label{narasi-90-detik-elevator-pitch}

``Saya seorang profesor dan elder yang merancang ekosistem belajar dan
pelayanan berbasis nilai. Karunia saya mengajar, membimbing, dan
memimpin dengan pendekatan sistem: mengubah konsep besar menjadi peta,
rubrik, dan alat yang membuat orang bertumbuh. Hati saya pada pendidikan
bermakna, kesejahteraan lansia dan keluarga, serta penguatan jemaat.
Dengan pengalaman lintas riset, kurikulum, dan pelayanan, saya
menghubungkan ilmu, iman, dan aksi. Target saya sederhana: menghadirkan
karya dan komunitas yang saling menguatkan---di kelas, di gereja, dan di
masyarakat---agar lebih banyak orang hidup berkualitas, berpengharapan,
dan siap melayani.''

\begin{center}\rule{0.5\linewidth}{0.5pt}\end{center}

\section{8) Service‑Fit Map (Tempat Saya Paling
Berdampak)}\label{servicefit-map-tempat-saya-paling-berdampak}

\begin{itemize}
\tightlist
\item
  \textbf{Kampus:} perancangan kurikulum \& rubrik; mentorship riset;
  otomasi pipeline belajar; kuliah \& orasi.
\item
  \textbf{Jemaat:} pembinaan rohani \& khotbah; fasilitasi sarasehan;
  program lansia/keluarga (GRACE).
\item
  \textbf{Riset‑Inovasi:} desain platform nilai‑ciptakan (PSKVE);
  publikasi; konsorsium kolaborasi.
\item
  \textbf{Kreasi Naratif:} kisah/lirik sebagai media edukasi \&
  penguatan batin.
\end{itemize}

\begin{center}\rule{0.5\linewidth}{0.5pt}\end{center}

\section{9) Evidences (Artefak \&
Tautan)}\label{evidences-artefak-tautan}

\begin{quote}
Ganti tanda {[} {]} dengan tautan/berkas Anda.
\end{quote}

\begin{itemize}
\tightlist
\item[$\square$]
  Silabus \& rubrik II‑2100 / EL2007
\item[$\square$]
  Modul/website kelas \& bank soal
\item[$\square$]
  Khotbah/renungan \& materi sarasehan
\item[$\square$]
  Publikasi/SLR \& proposal riset (GRACE, dsb.)
\item[$\square$]
  Novel/lirik \& materi kreatif
\item[$\square$]
  Pipeline otomasi (Quarto/GitHub/Obsidian)
\end{itemize}

\begin{center}\rule{0.5\linewidth}{0.5pt}\end{center}

\section{10) Rencana Aksi 90 Hari
(SMART)}\label{rencana-aksi-90-hari-smart}

\begin{enumerate}
\def\labelenumi{\arabic{enumi}.}
\tightlist
\item
  \textbf{Rampungkan halaman UTS (KIPP/All‑About‑Me) end‑to‑end.}
  \emph{Outcome:} semua tugas berisi bukti + rubrik; \emph{Due:} T‑14
  hari.
\item
  \textbf{Mentor 3 tim mahasiswa menyusun artefak bernilai.}
  \emph{Outcome:} 3 proyek dengan metrik dampak; \emph{Due:} T‑45 hari.
\item
  \textbf{Pilot GRACE micro‑service di jemaat.} \emph{Outcome:} 1
  layanan kecil terukur (mis. pendampingan lansia); \emph{Due:} T‑90
  hari.
\item
  \textbf{Publikasi ringkas (working paper) integrasi
  iman--ilmu--nilai.} \emph{Outcome:} 1 naskah pra‑cetak; \emph{Due:}
  T‑75 hari.
\end{enumerate}

\begin{center}\rule{0.5\linewidth}{0.5pt}\end{center}

\section{11) SHAPE ↔ CPMK (Interpersonal \& Public
Communication)}\label{shape-cpmk-interpersonal-public-communication}

\begin{itemize}
\tightlist
\item
  \textbf{Self‑awareness \& refleksi (CPMK‑S):} dituangkan pada Piagam
  Diri \& Narasi 90 detik.
\item
  \textbf{Empati \& komunikasi etis (CPMK‑E):} Shepherding/Exhortation →
  khotbah, mentoring, review berempati.
\item
  \textbf{Storytelling \& presentasi (CPMK‑P):} Teaching + kreasi
  naratif → kuliah, cerita, lirik.
\item
  \textbf{Kolaborasi \& kepemimpinan (CPMK‑K):}
  Leadership/Administration → proyek riset/komunitas terukur.
\end{itemize}

\begin{center}\rule{0.5\linewidth}{0.5pt}\end{center}

\section{12) Self‑Assessment Rubrik UTS‑4 (isi
skormu)}\label{selfassessment-rubrik-uts4-isi-skormu}

\begin{longtable}[]{@{}
  >{\raggedright\arraybackslash}p{(\linewidth - 6\tabcolsep) * \real{0.3382}}
  >{\raggedright\arraybackslash}p{(\linewidth - 6\tabcolsep) * \real{0.4412}}
  >{\raggedleft\arraybackslash}p{(\linewidth - 6\tabcolsep) * \real{0.1471}}
  >{\raggedright\arraybackslash}p{(\linewidth - 6\tabcolsep) * \real{0.0735}}@{}}
\toprule\noalign{}
\begin{minipage}[b]{\linewidth}\raggedright
Kriteria
\end{minipage} & \begin{minipage}[b]{\linewidth}\raggedright
Deskripsi
\end{minipage} & \begin{minipage}[b]{\linewidth}\raggedleft
Skor (1--5)
\end{minipage} & \begin{minipage}[b]{\linewidth}\raggedright
Bukti
\end{minipage} \\
\midrule\noalign{}
\endhead
\bottomrule\noalign{}
\endlastfoot
Kelengkapan SHAPE & S‑H‑A‑P‑E jelas \& terisi & & \\
Koherensi Piagam Diri & misi‑nilai‑peran konsisten & & \\
Narasi 90 detik & ringkas, kuat, mengundang aksi & & \\
Evidence \& Aksi 90 hari & tautan bukti \& rencana SMART & & \\
\end{longtable}

\textbf{Total (maks 20):} {[} {]} \textbf{Tingkat:} ☐ A (≥85\%) ☐ B
(70--84\%) ☐ C (60--69\%) ☐ D (50--59\%) ☐ E (\textless50\%)

\begin{center}\rule{0.5\linewidth}{0.5pt}\end{center}

\section{13) Versi Ultra‑Ringkas (≤140
kata)}\label{versi-ultraringkas-140-kata}

``Saya profesor \& elder dengan karunia mengajar, membimbing, dan
memimpin secara sistemik. Hati saya pada pendidikan bernilai,
kesejahteraan lansia/keluarga (GRACE), dan penguatan jemaat. Kemampuan
saya merancang kurikulum, rubrik, dan alat otomasi belajar; menulis
ilmiah \& kreatif; serta menggerakkan kolaborasi. Pengalaman saya di
kampus, gereja, riset, dan kreasi konten mengajarkan integrasi
iman--ilmu--aksi. Misi saya menghadirkan ekosistem yang memerdekakan: di
kelas melalui pembelajaran bermakna; di jemaat melalui pelayanan kasih
yang terukur; dan di masyarakat melalui inovasi yang adil. Target 90
hari: menuntaskan artefak UTS, mementori 3 tim mahasiswa, memulai
layanan mikro GRACE, dan menerbitkan naskah ringkas.''

\section{Piagam Diri --- Armein Z. R.
Langi}\label{piagam-diri-armein-z.-r.-langi}

\textbf{Pernyataan Misi} Saya adalah insinyur-pendidik dan penulis yang
menyalakan sukacita belajar, menumbuhkan empati, dan merancang sistem
cerdas yang memuliakan Tuhan serta meningkatkan kualitas hidup keluarga,
kampus, dan komunitas. (Struktur mengikuti kerangka \emph{My
SHAPE}---Piagam Diri 1-halaman. )

\textbf{S --- Signature Strengths (inti kekuatan khas)} Humor,
Spiritualitas, Kreativitas, Suka Belajar, Keingintahuan, Pandangan
(wisdom/perspective), Bersyukur, Keadilan, Kecerdasan Sosial, Kejujuran,
Kepemimpinan. (Sumber: VIA Character Strengths Profile, 13 Okt 2025. )

\textbf{H --- Heart (nilai \& panggilan)} Empati sebagai kecerdasan
tertinggi; kebaikan lebih utama daripada sekadar pintar; pencarian ``The
True Reality''; sukacita hidup yang mengasihi; keluarga sebagai
ekosistem kasih. (Disimpulkan dari tulisan-tulisan Anda di blog:
\emph{Empati: Kecerdasan Tertinggi}; \emph{On Being Nice}; \emph{The
Truth, The True Reality}; tagline blog; catatan keluarga.
(\href{https://ii-2100.github.io/all-about-me/}{Armein Z. R. Langi in
the City of Eden}))

\textbf{A --- Aptitudes \& Acquired Skills (bakat \& keterampilan
kunci)} Perancangan \& penelitian sistem/komputasi (speech compression,
FPGA), rekayasa \& kurikulum, kepemimpinan akademik, penulisan \&
penceritaan, fasilitasi pembelajaran, sistem \& organisasi. (Contoh
teknis: riset speech compression \& desain kontrol prosesor pada awal
karier.
(\href{https://ii-2100.github.io/all-about-me/My_Song_for_You/index.html}{Armein
Z. R. Langi in the City of Eden}))

\textbf{P --- Personality (gaya kerja yang menonjol)}
Reflektif-analitis, empatik-inklusif, visioner, pembelajar antusias,
kolaboratif; berpihak pada keadilan \& integritas. (Disintesis dari pola
kekuatan VIA dan tema tulisan Anda. )

\textbf{E --- Experiences (jejak pembentuk identitas)}

\begin{itemize}
\tightlist
\item
  \textbf{Ketangguhan pribadi} --- ``The Child Who Learned to Walk at
  the Disneyland'': ketekunan, berjalan dalam dingin, terus melangkah
  menuju tujuan.
  (\href{https://ii-2100.github.io/all-about-me/My_Stories_for_You/index.html}{Armein
  Z. R. Langi in the City of Eden})
\item
  \textbf{Lompatan kompetensi awal} --- perjalanan riset: software
  speech compression jadi dalam 3 bulan; desain chip kontrol di Xilinx
  FPGA; menulis paper.
  (\href{https://ii-2100.github.io/all-about-me/My_Song_for_You/index.html}{Armein
  Z. R. Langi in the City of Eden})
\item
  \textbf{Keluarga \& komunitas} --- keluarga besar sebagai sumber
  nilai, pelayanan, dan sukacita.
  (\href{https://ii-2100.github.io/all-about-me/My_Shapes/index.html}{Armein
  Z. R. Langi in the City of Eden})
\item
  \textbf{Standar keunggulan} --- sensibilitas benchmarking sains \&
  pendidikan (refleksi tentang Caltech).
  (\href{https://ii-2100.github.io/all-about-me/My_Personal_Reviews/index.html}{Armein
  Z. R. Langi in the City of Eden})
\end{itemize}

\textbf{Janji Praktis (Operating Principles)}

\begin{enumerate}
\def\labelenumi{\arabic{enumi}.}
\tightlist
\item
  \emph{People first with empathy} • 2) \emph{Truth-seeking with
  humility} • 3) \emph{Design for value \& justice} • 4) \emph{Teach
  what I practice, practice what I teach} • 5) \emph{Joyful learning,
  faithful living}. (Kerangka dan cara merangkum diadaptasi dari
  \emph{My SHAPE Toolkit}. )
\end{enumerate}

\begin{center}\rule{0.5\linewidth}{0.5pt}\end{center}

\section{Narasi Diri (versi 90 detik)}\label{narasi-diri-versi-90-detik}

Saya Armein---insinyur, pendidik, dan penulis---yang percaya bahwa
pengetahuan hanya bermakna bila melahirkan kasih dan keadilan. Kekuatan
saya adalah \textbf{spiritualitas yang membumi, kreativitas rekayasa,
dan kegembiraan belajar tanpa henti}, yang saya pakai untuk menyalakan
semangat orang lain.

Perjalanan saya ditempa oleh pengalaman yang mengajarkan
\textbf{ketekunan}---mulai dari ``berjalan dalam dingin'' hingga tuntas
menyelesaikan riset komputasi dan merancang sistem sejak awal karier.
Keluarga dan komunitas menjadi ekosistem kasih tempat saya belajar bahwa
\textbf{kebaikan lebih tinggi nilainya daripada sekadar pintar} dan
\textbf{empati adalah kecerdasan tertinggi}.
(\href{https://ii-2100.github.io/all-about-me/My_Stories_for_You/index.html}{Armein
Z. R. Langi in the City of Eden})

Ke depan, saya ingin terus \textbf{mendesain lingkungan belajar dan
sistem cerdas} yang memuliakan Tuhan dan membawa berkat
nyata---membentuk insan pembelajar yang jujur, adil, dan penuh
syukur---seraya menjaga sukacita: \emph{joy of loving and exciting
life}.
(\href{https://ii-2100.github.io/all-about-me/My_Stories_for_You/index.html}{Armein
Z. R. Langi in the City of Eden})

\begin{center}\rule{0.5\linewidth}{0.5pt}\end{center}

\section{Narasi Diri (versi panjang, 3--5
paragraf)}\label{narasi-diri-versi-panjang-35-paragraf}

\textbf{Kini.} Saya mengabdikan diri sebagai insinyur-pendidik yang
merancang pengalaman belajar dan sistem cerdas agar manusia bertumbuh
utuh: cakap teknis, peka nurani, dan gembira belajar. Kekuatan
saya---spiritualitas, kreativitas, suka belajar, keingintahuan,
perspektif, keadilan, dan kepemimpinan---mengarahkan cara saya memimpin,
mengajar, dan menulis.

\textbf{Dulu---titik balik.} Saya belajar bahwa langkah kecil yang
konsisten mengalahkan rintangan besar: berjalan sendirian dalam
dingin---secara harfiah dan metaforis---membentuk ketahanan batin. Di
laboratorium, saya menuntaskan perangkat lunak \textbf{speech
compression} dalam waktu singkat dan merancang \textbf{control unit}
berbasis FPGA, lalu menulis paper pertama---momen yang mengajarkan
disiplin, standar mutu, dan keberanian intelektual.
(\href{https://ii-2100.github.io/all-about-me/My_Stories_for_You/index.html}{Armein
Z. R. Langi in the City of Eden})

\textbf{Nilai yang saya pegang.} Saya memilih \textbf{kebaikan} di atas
sekadar \textbf{kepintaran}, menempatkan \textbf{empati} sebagai
kecerdasan tertinggi, dan mengejar \textbf{kebenaran sebagai realitas
yang sesungguhnya}. Keluarga besar meneguhkan panggilan itu---rumah
sebagai laboratorium kasih dan keadilan.
(\href{https://azrl.wordpress.com/2010/01/09/on-being-nice/?utm_source=chatgpt.com}{Armein
Z. R. Langi in the City of Eden})

\textbf{Ke depan.} Dengan kerangka \textbf{My SHAPE}, saya terus
menyelaraskan kekuatan, nilai, keterampilan, gaya kerja, dan pengalaman
untuk merancang program, riset, dan karya tulis yang berdampak---agar
teknologi menjadi pelayan kemanusiaan, dan setiap ruang belajar menjadi
\emph{City of Eden}: tempat sukacita, kasih, dan keunggulan bertemu.

\bookmarksetup{startatroot}

\chapter{UTS-5 My Personal Reviews}\label{uts-5-my-personal-reviews}

Berikut cara saya melakukan review: mengguan chatGPT, saya mengattach
\href{skor_uts.pdf}{file promt ChatGPT}, disertai perintah :``self
assess uts-1 sanpai uts-5 dari URL
`https://ii-2100.github.io/all-about-me/'\,''

ChatGPT melakukan self-assessment UTS-1 s.d. UTS-5 langsung dari laman
yang Anda berikan dan menilai memakai rubrik tugas UTS (skala 1--5 per
kriteria). Rekap skor siap diunduh sebagai CSV:
\href{sandbox:/mnt/data/UTS_self_assessment.csv}{Download CSV
ringkasan}.

\bookmarksetup{startatroot}

\chapter{Hasil Self-Assessment UTS (URL:
ii-2100.github.io/all-about-me)}\label{hasil-self-assessment-uts-url-ii-2100.github.ioall-about-me}

\section{Identifikasi}\label{identifikasi}

\begin{itemize}
\tightlist
\item
  Nama \& NIM penulis: \textbf{Armein Z. R. Langi -- 131902360} (tertera
  di halaman depan portofolio).
  (\href{https://ii-2100.github.io/all-about-me/}{II 2100})
\item
  Penilai: \textbf{Self-assessment (Armein Z. R. Langi)}
\item
  Catatan cakupan: halaman beranda memuat ``About Me''; navigasi ke ``My
  Songs for You'', ``My Stories for You'', ``My Shapes'', dan ``My
  Personal Reviews'' tersedia.
  (\href{https://ii-2100.github.io/all-about-me/}{II 2100})
\end{itemize}

\section{Tinjauan Umum}\label{tinjauan-umum}

\begin{itemize}
\tightlist
\item
  \textbf{UTS-1 (All About Me)} hadir di beranda (``Selamat Berjumpa /
  About Me''). Isi memperkenalkan identitas dan latar personal secara
  padat. (\href{https://ii-2100.github.io/all-about-me/}{II 2100})
\item
  \textbf{UTS-2 (My Songs for You)} memuat judul karya dan tautan audio,
  namun lirik/isi tidak ditampilkan di halaman (file audio tidak bisa
  saya akses dari sini), sehingga penilaian konten terbatas pada
  kelengkapan presentasi.
  (\href{https://ii-2100.github.io/all-about-me/My_Song_for_You/index.html}{II
  2100})
\item
  \textbf{UTS-3 (My Stories for You)} berisi tautan ke beberapa cerita;
  konten naratif kuat, terutama pada kisah keluarga ``Gaun Pengantin
  Gladys'' serta kisah inspiratif lainnya.
  (\href{https://ii-2100.github.io/all-about-me/My_Stories_for_You/index.html}{II
  2100})
\item
  \textbf{UTS-4 (My SHAPE)} saat ini halaman kosong/tanpa isi, hanya
  navigasi.
  (\href{https://ii-2100.github.io/all-about-me/My_Shapes/index.html}{II
  2100})
\item
  \textbf{UTS-5 (My Personal Reviews)} berisi metode/tautan panduan
  review, namun belum ada contoh \textbf{review personal} yang lengkap
  terhadap sebuah pesan/teks.
  (\href{https://ii-2100.github.io/all-about-me/My_Personal_Reviews/index.html}{II
  2100})
\end{itemize}

\begin{center}\rule{0.5\linewidth}{0.5pt}\end{center}

\section{Tinjauan Spesifik + Skor
(1--5)}\label{tinjauan-spesifik-skor-15}

\subsection{UTS-1 --- All About Me (di
beranda)}\label{uts-1-all-about-me-di-beranda}

\textbf{Skor per kriteria:} Orisinalitas \textbf{3}, Keterlibatan
\textbf{3}, Humor \textbf{1}, Wawasan/Insight \textbf{3} → \textbf{Total
10/20 (50\%)}. \textbf{Alasan singkat:} Perkenalan diri faktual dan
jelas, namun gaya masih informatif ketimbang naratif (minim humor dan
moment ``hook''). (\href{https://ii-2100.github.io/all-about-me/}{II
2100}) \textbf{Saran perbaikan:} Buka dengan anekdot pribadi (1--2
paragraf) yang ``mengikat'' (mis. titik balik karier/keluarga),
tambahkan satu momen humor ringan, lalu tutup dengan refleksi singkat
tentang nilai/visi diri agar aspek insight naik.

\subsection{UTS-2 --- My Songs for You}\label{uts-2-my-songs-for-you-1}

\textbf{Skor per kriteria:} Orisinalitas \textbf{2}, Keterlibatan
\textbf{2}, Humor \textbf{1}, Inspirasi \textbf{2} → \textbf{Total 7/20
(35\%)}. \textbf{Alasan singkat:} Halaman menampilkan judul lagu ``River
in my Mind'', ``Heaven on Earth'', namun tanpa lirik/cerita di balik
lagu sehingga sulit menilai aspek pesan, humor, dan inspirasi.
(\href{https://ii-2100.github.io/all-about-me/My_Song_for_You/index.html}{II
2100}) \textbf{Saran perbaikan:} Tambahkan lirik lengkap, 1 paragraf
cerita proses kreatif, dan 2--3 kalimat ``pesan untukmu'' agar inspirasi
terbaca; sertakan player/tautan yang dapat diputar langsung + fallback
transkrip.

\subsection{UTS-3 --- My Stories for
You}\label{uts-3-my-stories-for-you-1}

\textbf{Skor per kriteria:} Orisinalitas \textbf{5}, Keterlibatan
\textbf{5}, Pengembangan Narasi \textbf{4}, Inspirasi \textbf{5} →
\textbf{Total 19/20 (95\%)}. \textbf{Alasan singkat:} Cerita ``Gaun
Pengantin Gladys'' dkk sangat personal, emosional, dan inspiratif; ritme
bertutur hidup serta detail situasional kuat (konten ramu unsur
penebusan/keluarga/iman).
(\href{https://ii-2100.github.io/all-about-me/My_Stories_for_You/index.html}{II
2100}) \textbf{Saran perbaikan:} Tambah ``lead'' 2--3 kalimat yang
merangkum pesan kunci tiap cerita; akhiri dengan ajakan/refleksi 1--2
kalimat agar resonansi ke pembaca makin jelas.

\subsection{UTS-4 --- My SHAPE}\label{uts-4-my-shape}

\textbf{Skor per kriteria:} Orisinalitas \textbf{1}, Keterlibatan
\textbf{1}, Pengembangan \textbf{1}, Inspirasi \textbf{1} →
\textbf{Total 4/20 (20\%)}. \textbf{Alasan singkat:} Konten belum
tersedia.
(\href{https://ii-2100.github.io/all-about-me/My_Shapes/index.html}{II
2100}) \textbf{Saran perbaikan (prioritas):}

\begin{enumerate}
\def\labelenumi{\arabic{enumi}.}
\tightlist
\item
  Tampilkan rekap SHAPE (Spiritual gifts, Heart, Abilities, Personality,
  Experiences) dalam 1 tabel ringkas + 3--5 butir bukti;
\item
  Narasikan 2 pengalaman kunci (±150 kata/each) yang menunjukkan
  ``signature strengths'';
\item
  Tutup dengan 3 rencana aksi konkret (who/what/when).
\end{enumerate}

\subsection{UTS-5 --- My Personal
Reviews}\label{uts-5-my-personal-reviews-1}

\textbf{Skor per kriteria:} Pemahaman Konsep \textbf{2}, Analisis Kritis
\textbf{1}, Argumentasi (Logos) \textbf{1}, Etos \& Empati \textbf{2},
Rekomendasi \textbf{1} → \textbf{Total 7/25 (28\%)}. \textbf{Alasan
singkat:} Halaman berisi metode cara menilai, tetapi \textbf{belum ada}
contoh \textbf{review personal} yang lengkap terhadap sebuah pesan/teks
sehingga aspek analisis-argumentasi tak terbaca.
(\href{https://ii-2100.github.io/all-about-me/My_Personal_Reviews/index.html}{II
2100}) \textbf{Saran perbaikan:} Pilih 1 karya personal (mis.
UTS-1/2/3), tulis review 400--600 kata: ringkas pesan, nilai dengan
rubrik, berikan 2--3 bukti kutipan, evaluasi etos/empati, lalu tutup
dengan rekomendasi perbaikan spesifik.

\begin{center}\rule{0.5\linewidth}{0.5pt}\end{center}

\section{Rekap Skor (ringkas)}\label{rekap-skor-ringkas}

\begin{itemize}
\tightlist
\item
  \textbf{UTS-1:} 10/20 → \textbf{50\%}
\item
  \textbf{UTS-2:} 7/20 → \textbf{35\%}
\item
  \textbf{UTS-3:} 19/20 → \textbf{95\%}
\item
  \textbf{UTS-4:} 4/20 → \textbf{20\%}
\item
  \textbf{UTS-5:} 7/25 → \textbf{28\%}
\end{itemize}

CSV lengkap sudah saya siapkan untuk dokumentasi dan olah lanjut:
\href{sandbox:/mnt/data/UTS_self_assessment.csv}{Download CSV
ringkasan}.

\section{Langkah Perbaikan Cepat (prioritas 1
minggu)}\label{langkah-perbaikan-cepat-prioritas-1-minggu}

\begin{enumerate}
\def\labelenumi{\arabic{enumi}.}
\tightlist
\item
  \textbf{Lengkapi UTS-4 (My SHAPE)} sesuai saran di atas;
\item
  \textbf{Tambahkan lirik + sinopsis} untuk tiap lagu di UTS-2 agar bisa
  dinilai isi/inspirasinya;
\item
  \textbf{Tulis 1 review personal penuh} (UTS-5) memakai rubrik (pilih
  salah satu karya Anda sendiri sebagai objek);
\item
  Revisi UTS-1 dengan pembuka anekdot + refleksi nilai diri.
\end{enumerate}

Jika Anda mau, saya bisa bantu merapikan UTS-4 (tabel SHAPE + narasi)
dan membuat kerangka cepat untuk \textbf{review} di UTS-5 dari salah
satu karya Anda.

\bookmarksetup{startatroot}

\chapter{UAS-1 My Concepts}\label{uas-1-my-concepts}

Mau hidup epik ? \href{lifestory.pdf}{Write your Life Story}

Apa itu berkonsep?

\url{https://youtu.be/QVfUlVBO80U?si=yM6q_rwV9rcDBbu7}

\bookmarksetup{startatroot}

\chapter{UAS-3 My Opinions}\label{uas-3-my-opinions}

SApa itu beropini? \href{BM\%20Opini.mp4}{Opini Berpengaruh}

Bagiamana menjaadi menarik? \href{./Interesting.mp4}{Menjadi Menarik}

\bookmarksetup{startatroot}

\chapter{UAS-3 My Innovations}\label{uas-3-my-innovations}

\bookmarksetup{startatroot}

\chapter{UAS-4 My Knowledge}\label{uas-4-my-knowledge}

Cara saya mengkomunikasikan sebuah pengatahuan sebagai petunjuk bagi
orang lain 1) saya tulis
\href{Rekomendasi\%20Presentasi\%20Efektif(Contoh\%20Makalah).pdf}{makalah
sebagai bahan utama} 2) lalu saya buat
\href{Contoh\%20Transkrip\%20Presentasi.pdf}{transkrip ucapan lisan} 3)
kemudian saya kembangkan
\href{Rekomendasi\%20Presentasi\%20(Contoh\%20Slides).pdf}{slide
pendukung trnsskrip} 4) lalu saya memproduksivideo audio visual
\url{https://youtu.be/ZbghfMvnPZc} \url{https://youtu.be/ZbghfMvnPZc}

\bookmarksetup{startatroot}

\chapter{UAS-5 My Professional
Reviews}\label{uas-5-my-professional-reviews}

Untuk melAkukan review, seperti pada
\href{../My_Personal_Reviews/Doc.5.Mengevaluasi-Esai-Berdasarkan-Rubrik.pdf}{pendekatan
AI}, kita membutuhkan rubrik

\bookmarksetup{startatroot}

\chapter{Summary}\label{summary}

In summary, this book has no content whatsoever.

\bookmarksetup{startatroot}

\chapter*{References}\label{references}
\addcontentsline{toc}{chapter}{References}

\markboth{References}{References}

\phantomsection\label{refs}




\end{document}
